\documentclass[a4paper]{article}

\usepackage[margin = 3cm]{geometry}
\usepackage{amsmath, amssymb}
\usepackage[T1]{fontenc}
\usepackage{listings}
\usepackage{xcolor}

\lstset{%
    %backgroundcolor=\color{yellow!20},%
    basicstyle=\ttfamily%
    }%

\begin{document}
 
\title{VEMLAB User Guide}
\author{Massimo Frittelli}

\maketitle

\section*{Introduction}
VEMLAB is an object-oriented MATLAB library for the computation of the VEM matrices of lowest order ($k=1$) of the PDE problem
\begin{equation}
\frac{\partial u}{\partial t} - \Delta u + u = 0,
\end{equation}
in 2D and 3D, on quite general polygonal/polyhedral meshes. The library relies on the following assumptions on the mesh:
\begin{itemize}
\item in 2D, every (polygonal) element is star-shaped w.r.t. at least one point.
\item in 3D, every (polyhedral) element is star-shaped w.r.t. at least one point and so are all of its faces.
\end{itemize}

\section{The class \texttt{element2d}}
The class \texttt{element2d} represents a polygonal element in 2D with \texttt{NVert} vertexes that is star-shaped w.r.t. at least one point. To create an instance of the class, use the following constructor

\begin{lstlisting}
	obj = element2d(P, P0);
\end{lstlisting}
%
where
\begin{itemize}
\item \texttt{P} is a $\texttt{NVert} \times 3$ matrix whose rows are the coordinates of the ordered vertexes. 
\item \texttt{P0}, of size $1\times 3$, is a point w.r.t. which the element is star-shaped.
\end{itemize}

\noindent
Upon initialisation, the object stores \texttt{P} and \texttt{P0} and automatically computes several other \texttt{properties} of the element:

\begin{lstlisting}
	properties(SetAccess = private)
		P(:,3) double
		P0(1,3) double 
		NVert(1,1) double
		Area(1,1) double 
		OrientedArea(1,3) double
		Centroid (1,3) double
		Diameter(1,1) double
		K(:,:) double
		M(:,:) double
	end
\end{lstlisting}

\noindent
that can be queried from the object. In the above:
\begin{itemize}
\item \texttt{NVert} is the number of vertexes
\item \texttt{Area} is the surface area of the element
\item \texttt{OrientedArea} is a vector orthogonal to the element whose modulus is the element area
\item \texttt{Centroid} is the centroid of the element
\item \texttt{Diameter} is the diameter of the element
\item \texttt{K} is the local stiffness matrix
\item \texttt{M} is the local mass matrix
\end{itemize}

\noindent
The usage of \texttt{element2d} will be demonstrated later on.

\section{The class \texttt{element3d}}
The class \texttt{element3d} represents a polygonal element in 3D with \texttt{NVert} vertexes that is star-shaped w.r.t. at least one point and whose faces fulfill the same property.  To create an instance of the class, use the following constructor

\begin{lstlisting}
	obj = element3d(Faces, P, P0);
\end{lstlisting}
%
where 
\begin{itemize}
\item \texttt{Faces} is a $\texttt{NFaces} \times 1$ array of \texttt{element2d} representing the faces
\item \texttt{P} is a $\texttt{NVert} \times 3$ matrix whose rows are the coordinates of the vertexes
\item \texttt{P0} is a point w.r.t. which the element is star-shaped.
\end{itemize}
We remark that, even if the vertexes \texttt{P} are already contained in the \texttt{Faces}, the \texttt{property} \texttt{P} is still needed to specify vertex ordering. Upon initialisation, the object stores \texttt{Faces}, \texttt{P}, and \texttt{P0} and automatically computes several other public \texttt{properties} of the element:

\begin{lstlisting}
	properties(SetAccess = private)
		Faces(:,1) element2d
		P(:,3) double
		P0(1,3) double
		NVert(1,1) double
		NFaces(1,1) double
		Volume(1,1) double
		Centroid(1,3) double
		Diameter(1,1) double
		K(:,:) double
		M(:,:) double
	end
\end{lstlisting}

\noindent
that can be queried from the object. In the above:
\begin{itemize}
\item \texttt{NVert} is the number of vertexes and \texttt{NFaces} is the number of faces
\item \texttt{Volume}, \texttt{Centroid} and \texttt{Diameter} are self-explanatory
\item \texttt{K} is the local stiffness matrix and \texttt{M} is the local mass matrix.
\end{itemize}

\noindent
The usage of \texttt{element3d} will be demonstrated later on.

\section{A worked example in 2D: the unit square}
Here we will show the usage of \texttt{element2d} to compute the local matrices of the unit square, thereby presenting the closed-form counterpart. Consider the unit square contained in the $xy$-plane:
\begin{equation}
F = \{(x,y,z) \in \mathbb{R}^3 | (x,y) \in [0,1]^2, \ z = 0\},
\end{equation}
which can be thought of as the polygon enclosed by the vertexes $(0, 0, 0)$, $(0, 1, 0)$, $(1,1, 0)$, and $(1, 0, 0)$.  Notice that node ordering affects the resulting matrices.  We start by computing the closed form of the VEM local mass and stiffness matrices of $F$ for the lowest order case $k=1$.

\noindent
As shown in \cite{hitchhikers}, the computation of the mass and stiffness matrices relies on three fundamental matrices:
\begin{itemize}
\item $B \in \mathbb{R}^{3\times\texttt{NVert}}$;
\item $D \in \mathbb{R}^{\texttt{NVert} \times 3}$;
\item $H \in\mathbb{R}^{3\times 3}$,
\end{itemize}
whose lengthy definitions we do not report here. With the above matrices in hand, the following matrices can be obtained:
\begin{itemize}
\item $G := BD \in\mathbb{R}^{3\times 3}$;
\item $\widetilde{G} := \left[\begin{array}{ccc}
0 & 0 & 0\\ 0 & 1 & 0 \\ 0 & 0 & 1
\end{array}\right] G \in\mathbb{R}^{3\times 3}$;
\item $\Pi^\nabla_* := G^{-1}B \in \mathbb{R}^{3\times\texttt{NVert}}$;
\item $\Pi^\nabla := D\Pi^\nabla_* \in \mathbb{R}^{\texttt{NVert}\times\texttt{NVert}}$.
\end{itemize}

\noindent
Finally, the local stiffness and mass matrices are given by
\begin{align}
&K = (\Pi^\nabla_*)^T \widetilde{G} \Pi^\nabla_* + (I-\Pi^\nabla)^T(I-\Pi^\nabla);\\
&M = (\Pi^\nabla_*)^T H \Pi^\nabla_* + \text{Area(F)}(I-\Pi^\nabla)^T(I-\Pi^\nabla).
\end{align}

\noindent
For the unit square $F$,  as shown in \cite{hitchhikers}, it holds that
\begin{align}
&B = \frac{1}{4}\left[\begin{array}{c c c c}
\ \ 1 & \ \ 1 & \ \ 1 & \ \ 1\\
-\sqrt{2} & \ \ \sqrt{2} & \ \ \sqrt{2} & -\sqrt{2}\\
-\sqrt{2} & -\sqrt{2} & \ \ \sqrt{2} & \ \ \sqrt{2}
\end{array}\right];\\
%
&D = \frac{1}{4}\left[\begin{array}{c c c}
4 & -\sqrt{2} & -\sqrt{2}\\
4 & \ \ \sqrt{2} & -\sqrt{2}\\
4 & \ \ \sqrt{2} & \ \ \sqrt{2}\\
4 & -\sqrt{2} & \ \ \sqrt{2}\\
\end{array}\right];\\
%
&H = \frac{1}{24}\left[\begin{array}{c c c}
24 & 0 & 0\\
0 & 1 & 0\\
0 & 0 & 1\\
\end{array}\right].
\end{align}

\noindent
It follows that
\begin{align}
G = \frac{1}{2}&\left[\begin{array}{c c c}
2 & 0 & 0\\
0 & 1 & 0\\
0 & 0 & 1\\
\end{array}\right],
\qquad 
\widetilde{G} = \frac{1}{2}\left[\begin{array}{c c c}
0 & 0 & 0\\
0 & 1 & 0\\
0 & 0 & 1\\
\end{array}\right];\\
%
\Pi^\nabla_* = \frac{1}{4}&\left[\begin{array}{c c c c}
\ \ 1 & \ \ 1 & \ \ 1 & \ \ 1\\
-2\sqrt{2} & \ \ 2\sqrt{2} & \ \ 2\sqrt{2} & -2\sqrt{2}\\
-2\sqrt{2} & -2\sqrt{2} & \ \ 2\sqrt{2} & \ \ 2\sqrt{2}
\end{array}\right];\\
%
\Pi^\nabla = \frac{1}{4}&\left[\begin{array}{c c c c}
\ \ 3 & \ \ 1 & -1 & \ \ 1\\
\ \ 1 & \ \ 3 & \ \ 1 & -1\\
-1 & \ \ 1 & \ \ 3 & \ \ 1\\
\ \ 1 & -1 & \ \ 1 & \ \ 3\\
\end{array}\right].
\end{align}
We finally obtain the local stiffness and mass matrices:
\begin{align}
\label{stiffness_2d}
K = \frac{1}{4}&\left[\begin{array}{c c c c}
\ \ 3 &  -1 & -1 &  -1\\
 -1 & \ \ 3 &  -1 & -1\\
-1 &  -1 & \ \ 3 & -1\\
 -1 & -1 & - 1 & \ \ 3\\
\end{array}\right];\\
%
\label{mass_2d}
M = \frac{1}{48}&\left[\begin{array}{c c c c}
 \ \ \ 17 & -9 & \ \ \ 13 &  -9\\
 -9 &  \ \ \ 17 &  -9 & \ \ \ 13\\
\ \ \ 13 &  -9 &  \ \ \ 17 & -9\\
 -9 & \ \ \ 13 & - 9 & \ \ \ 17\\
\end{array}\right].
\end{align}

\noindent
We now show the usage of \texttt{element2d} to compute the matrices $K$ and $M$ numerically. To this end, we need to create an object of class \texttt{element2d}. To call the constructor, we define the array \texttt{P}
\begin{lstlisting}
	P = [0 0 0; 0 1 0; 1 1 0; 1 0 0];
\end{lstlisting}
containing the vertexes of \texttt{F} and the array \texttt{P0} as
\begin{lstlisting}
	P0 = [.5 .5 0];
\end{lstlisting}
because $F$ is star-shaped w.r.t. \texttt{P0}. Notice that, since $F$ is convex,  \texttt{P0} can be chosen as any point in $F$, even a vertex. We are ready to create the object:
\begin{lstlisting}
	F = element2d(P, P0)
\end{lstlisting}
Because there is no semicolon in the above command, the following output appears in the command window:
\begin{lstlisting}
E1 = 

  element2d with properties:

               P: [4x3 double]
              P0: [0.5000 0.5000 0]
           NVert: 4
            Area: 1
    OrientedArea: [0 0 -1]
        Centroid: [0.5000 0.5000 0]
        Diameter: 1.4142
               K: [4x4 double]
               M: [4x4 double]
\end{lstlisting}

\noindent
Accidentally, the \texttt{Centroid} coincides with \texttt{P0}. By querying the stiffness and mass matrices of \texttt{F} (with \texttt{format rat} for better readability), we can see the outputs

\begin{lstlisting}
>> E1.K

ans =

       3/4        -1/4        -1/4        -1/4     
      -1/4         3/4        -1/4        -1/4     
      -1/4        -1/4         3/4        -1/4     
      -1/4        -1/4        -1/4         3/4     
      
>> E1.M

ans =

      17/48       -3/16       13/48       -3/16    
      -3/16       17/48       -3/16       13/48    
      13/48       -3/16       17/48       -3/16    
      -3/16       13/48       -3/16       17/48    
\end{lstlisting}

\noindent
which agree with \eqref{stiffness_2d}-\eqref{mass_2d}.


\section{A worked example in 3D: the unit cube}
Here we will show the usage of \texttt{element3d} to compute the local matrices of the unit cube $E = [0,1]^3$, thereby presenting the closed-form counterpart.  Because vertex ordering is reflected in the resulting matrices, we order the vertexes as follows:
\begin{equation}
\label{cube_vertex_ordering}
(0, 0,0)\ 
(0,0, 1)\ 
(0,1, 0)\ 
(0, 1,1)\ 
(1 ,0,0)\ 
(1 ,0,1)\ 
(1 ,1,0)\ 
(1 ,1,1).
\end{equation}

\noindent
We start by computing the closed form of the VEM local mass and stiffness matrices of $E$ for the lowest order case $k=1$.  As shown in \cite{hitchhikers}, the computation of the mass and stiffness matrices relies on three fundamental matrices, similarly to the 2D case:
\begin{itemize}
\item $B \in \mathbb{R}^{4\times\texttt{NVert}}$;
\item $D \in \mathbb{R}^{\texttt{NVert} \times 4}$;
\item $H \in\mathbb{R}^{4\times 4}$,
\end{itemize}
whose lengthy definitions we do not report here. With the above matrices in hand, the following matrices can be obtained:
\begin{itemize}
\item $G := BD \in\mathbb{R}^{4\times 4}$;
\item $\widetilde{G} := \left[\begin{array}{cccc}
0 & 0 & 0 & 0\\ 0 & 1 & 0 & 0\\ 0 & 0 & 1 & 0\\ 0 & 0 & 0 & 1
\end{array}\right] G \in\mathbb{R}^{4\times 4}$;
\item $\Pi^\nabla_* := G^{-1}B \in \mathbb{R}^{4\times\texttt{NVert}}$;
\item $\Pi^\nabla := D\Pi^\nabla_* \in \mathbb{R}^{\texttt{NVert}\times\texttt{NVert}}$.
\end{itemize}

\noindent
Finally, the local stiffness and mass matrices are given by
\begin{align}
&K = (\Pi^\nabla_*)^T \widetilde{G} \Pi^\nabla_* + \text{Diam}(E)(I-\Pi^\nabla)^T(I-\Pi^\nabla);\\
&M = (\Pi^\nabla_*)^T H \Pi^\nabla_* + \text{Volume(F)}(I-\Pi^\nabla)^T(I-\Pi^\nabla).
\end{align}

\noindent
For the unit cube $E$,  it is possible to show that
\begin{align}
B = \frac{1}{8\sqrt{3}}&\left[\begin{array}{c c c c c c c c}
\sqrt{3} & \sqrt{3} & \sqrt{3} & \sqrt{3} & \sqrt{3} & \sqrt{3} & \sqrt{3} & \sqrt{3}\\
-2 & -2 & -2 & -2 & \ \ 2 & \ \ 2 & \ \ 2 & \ \ 2\\
-2 & -2 & \ \ 2 & \ \ 2 & -2 & -2 & \ \ 2 & \ \ 2\\
-2 & \ \ 2 & -2 & \ \ 2 & -2 & \ \ 2 & -2 & \ \ 2\\
\end{array}\right];\\
%
D = \frac{1}{2\sqrt{3}}&\left[\begin{array}{c c c c}
2\sqrt{3} & -1 & -1 & -1\\
2\sqrt{3} & -1 & -1 & \ \ 1\\
2\sqrt{3} & -1 & \ \ 1 & -1\\
2\sqrt{3} & -1 & \ \ 1 & \ \ 1\\
2\sqrt{3} & \ \ 1 & -1 & -1\\
2\sqrt{3} & \ \ 1 & -1 & \ \ 1\\
2\sqrt{3} & \ \ 1 & \ \ 1 & -1\\
2\sqrt{3} & \ \ 1 & \ \ 1 & \ \ 1
\end{array}\right];\\
%
H = \frac{1}{36}&\left[\begin{array}{c c c c}
36 & 0 & 0 & 0\\
0 & 1 & 0 & 0\\
0 & 0 & 1 & 0\\
0 & 0 & 0 & 1
\end{array}\right].
\end{align}

\noindent
It follows that
\begin{align}
G = \frac{1}{3}&\left[\begin{array}{c c c c}
3 & 0 & 0 & 1\\
0 & 1 & 0 & 0\\
0 & 0 & 1 & 0\\
0 & 0 & 0 & 1
\end{array}\right],
\qquad 
\widetilde{G} = \frac{1}{3}\left[\begin{array}{c c c c}
0 & 0 & 0 & 1\\
0 & 1 & 0 & 0\\
0 & 0 & 1 & 0\\
0 & 0 & 0 & 1
\end{array}\right];\\
%
\Pi^\nabla_* =\frac{1}{8}&\left[\begin{array}{c c c c c c c c}
1 & 1 & 1 & 1 & 1 & 1 & 1 & 1\\
-2\sqrt{3} & -2\sqrt{3} & -2\sqrt{3} & -2\sqrt{3} & \ \ 2\sqrt{3} & \ \ 2\sqrt{3} & \ \ 2\sqrt{3} & \ \ 2\sqrt{3}\\
-2\sqrt{3} & -2\sqrt{3} & \ \ 2\sqrt{3} & \ \ 2\sqrt{3} & -2\sqrt{3} & -2\sqrt{3} & \ \ 2\sqrt{3} & \ \ 2\sqrt{3}\\
-2\sqrt{3} & \ \ 2\sqrt{3} & -2\sqrt{3} & \ \ 2\sqrt{3} & -2\sqrt{3} & \ \ 2\sqrt{3} & -2\sqrt{3} & \ \ 2\sqrt{3}\\
\end{array}\right];\\
%
\Pi^\nabla = \frac{1}{4}&\left[\begin{array}{c c c c c c c c}
2 & 1 & 1 & 0 & 1 & 0 & 0 & \!\!\!\! -1\\
1 & 2 & 0 & 1 & 0 & 1 & \!\!\!\! -1 & 0\\
1 & 0 & 2 & 1 & 0 & \!\!\!\! -1 & 1 & 0\\
0 & 1 & 1 & 2 & \!\!\!\! -1 & 0 & 0 & 1\\
1 & 0 & 0 & \!\!\!\! -1 & 2 & 1 & 1 & 0\\
0 & 1 & \!\!\!\! -1 & 0 & 1 & 2 & 0 & 1\\
0 & \!\!\!\! -1 & 1 & 0 & 1 & 0 & 2 & 1\\
\!\!\!\! -1 & 0 & 0 & 1 & 0 & 1 & 1 & 2
\end{array}\right].
\end{align}

\noindent
We finally obtain the local stiffness and mass matrices:
\begin{align}
\label{stiffness_3d}
\begin{split}
K = \frac{1}{16}&\left[\begin{array}{c c c c c c c c}
\ \  3        &   \ \     1        &  \ \      1      &       -1       &    \ \     1       &      -1      &       -1            & -3       \\
  \ \       1          &  \ \    3      &       -1       &    \ \     1     &        -1         &   \ \    1        &     -3        &     -1       \\
   \ \      1       &      -1     &    \ \       3       &    \ \     1        &     -1       &      -3        &  \ \      1      &       -1       \\
      -1     &      \ \     1      &      \ \    1      &      \ \    3       &      -3      &       -1       &      -1      &     \ \     1       \\
   \ \      1        &     -1        &     -1     &        -3       &    \ \     3       &   \ \      1        &    \ \    1       &     -1       \\
      -1       &    \ \     1       &      -3       &      -1       &   \ \      1       &    \ \     3      &       -1       &   \ \      1       \\
      -1     &        -3      &    \ \      1       &      -1      &    \ \     1       &      -1       &   \ \      3       &    \ \     1       \\
      -3       &      -1       &      -1     &     \ \      1      &       -1      &   \ \       1     &      \ \     1        &   \ \     3    
\end{array}\right]\\
+ \frac{\sqrt{3}}{4}&\left[\begin{array}{c c c c c c c c}
 \ \  2      &       -1       &      -1            &  \ \  0       &      -1      &      \ \     0      &      \ \    0             & \ \   1       \\
      -1     &        \ \    2       &    \ \      0      &      -1      &       \ \    0       &      -1        &    \ \     1              &  \ \  0       \\
      -1      &     \ \      0       &    \ \      2     &        -1      &     \ \      0       &        \ \  1       &      -1   &       \ \         0       \\
    \ \      0     &        -1        &     -1      &      \ \     2        &    \ \     1       &        \ \  0        &    \ \     0 &           -1       \\
      -1      &    \ \       0        &    \ \     0        &    \ \    1        &     \ \    2       &      -1       &      -1       &    \ \      0       \\
     \ \     0     &        -1      &      \ \     1        &   \ \      0     &       -1       &        \ \  2        &     \ \    0  &           -1       \\
     \ \     0        &    \ \     1      &       -1      &     \ \      0      &      -1       &        \ \  0       &     \ \    2  &           -1       \\
     \ \     1       &      \ \    0       &   \ \       0     &        -1       &      \ \    0       &      -1        &     -1    &      \ \       2   \\
\end{array}\right];
\end{split}\\
%
\label{mass_3d}
M = \frac{1}{96}&\left[\begin{array}{c c c c c c c c}
 \ \ 51        &    -22       &     -22        &       1       &     -22        &       1        &    \ \    1        &    \ \   24     \\  
     -22        &      \ \ 51        &      1        &    -22     &      1        &    -22            &  \ \  24      &        1       \\
     -22      &        1       &    \ \    51    &        -22      &        1      &    \ \     24           & -22       &       1       \\
       1      &      -22       &     -22      &     \ \    51        &   \ \    24        &      1             & 1       &     -22       \\
     -22    &          1        &      1       &     \ \   24     &     \ \     51       &     -22           & -22       &       1       \\
       1      &      -22        &    \ \   24      &        1      &      -22       &     \ \   51             & 1   &         -22       \\
       1          &   \ \  24     &       -22       &       1      &      -22       &       1            &  \ \ 51       &     -22       \\
    \ \    24         &     1       &       1     &       -22      &        1      &      -22           & -22      &     \ \    51 
\end{array}\right].
\end{align}

\noindent
We now show the usage of \texttt{element3d} to compute the matrices $K$ and $M$ numerically. To this end, we need to create an object of class \texttt{element3d}. To call the constructor, we first need to create six instances of \texttt{element2d} representing the faces of $E$:

\begin{lstlisting}
P1 = [0 0 0; 0 1 0; 1 1 0; 1 0 0]; % bottom face
E1 = element2d(P1, sum(P1,1)/4);

P2 = [0 0 1; 0 1 1; 1 1 1; 1 0 1]; % top face
E2 = element2d(P2, sum(P2,1)/4);

P3 = [0 0 0; 0 1 0; 0 1 1; 0 0 1]; % back face
E3 = element2d(P3, sum(P3,1)/4);

P4 = [1 0 0; 1 1 0; 1 1 1; 1 0 1]; % front face
E4 = element2d(P4, sum(P4,1)/4);

P5 = [0 0 0; 1 0 0; 1 0 1; 0 0 1]; % left face
E5 = element2d(P5, sum(P5,1)/4);

P6 = [0 1 0; 1 1 0; 1 1 1; 0 1 1]; % right face
E6 = element2d(P6, sum(P6,1)/4);
\end{lstlisting}

\noindent
For each of the faces,  we have chosen \texttt{P0} as the midpoint of its vertexes for convenience, but of course other choices are possible since every face is convex and thus star-shaped w.r.t. every point of the face itself.  We are ready to create the \texttt{element3d}:
\begin{lstlisting}
P = unique([P1; P2; P3; P4; P5; P6],`rows');
E = element3d([E1;E2;E3;E4;E5;E6], P, sum(P,1)/8);
\end{lstlisting}
We have used the command \texttt{unique} to extract a set of all vertexes with no repetitions. MATLAB will sort the vertexes in \texttt{P} in ``increasing order'', that is as in \eqref{cube_vertex_ordering}.  Again, the \texttt{P0} is chosen as the midpoint of all vertexes for convenience. Let us have a look at the 3D element \texttt{E}:
\begin{lstlisting}
>> E

E = 

  element3d with properties:

       Faces: [6x1 element2d]
           P: [8x3 double]
          P0: [0.5000 0.5000 0.5000]
       NVert: 8
      NFaces: 6
      Volume: 1.0000
    Centroid: [0.5000 0.5000 0.5000]
    Diameter: 1.7321
           K: [8x8 double]
           M: [8x8 double]
\end{lstlisting}

\noindent
By querying the stiffness and mass matrices of \texttt{E}, we can see the outputs

\begin{lstlisting}
>> format
>> E.K

ans =

    1.0535  -0.3705  -0.3705  -0.0625  -0.3705  -0.0625  -0.0625   0.2455
   -0.3705   1.0535  -0.0625  -0.3705  -0.0625  -0.3705   0.2455  -0.0625
   -0.3705  -0.0625   1.0535  -0.3705  -0.0625   0.2455  -0.3705  -0.0625
   -0.0625  -0.3705  -0.3705   1.0535   0.2455  -0.0625  -0.0625  -0.3705
   -0.3705  -0.0625  -0.0625   0.2455   1.0535  -0.3705  -0.3705  -0.0625
   -0.0625  -0.3705   0.2455  -0.0625  -0.3705   1.0535  -0.0625  -0.3705
   -0.0625   0.2455  -0.3705  -0.0625  -0.3705  -0.0625   1.0535  -0.3705
    0.2455  -0.0625  -0.0625  -0.3705  -0.0625  -0.3705  -0.3705   1.0535
      
>> format rat
>> E.M

ans =

    17/32  -11/48  -11/48    1/96  -11/48    1/96    1/96    1/4
   -11/48   17/32    1/96  -11/48    1/96  -11/48    1/4     1/96
   -11/48    1/96   17/32  -11/48    1/96    1/4   -11/48    1/96    
     1/96  -11/48  -11/48   17/32    1/4     1/96    1/96  -11/48    
   -11/48    1/96    1/96    1/4    17/32  -11/48  -11/48    1/96    
     1/96  -11/48    1/4     1/96  -11/48   17/32    1/96  -11/48    
     1/96    1/4   -11/48    1/96  -11/48    1/96   17/32  -11/48    
     1/4     1/96    1/96  -11/48    1/96  -11/48  -11/48   17/32  
\end{lstlisting}

\noindent
which agree with \eqref{stiffness_3d}-\eqref{mass_3d} up to machine precision.


\bibliographystyle{plain}
\bibliography{bibliography}

\end{document}